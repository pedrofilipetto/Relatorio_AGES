\subsection{Sprint 2}

Com o início da segunda sprint dividimos, novamente, o time em 3 squads diferentes, porém foi alterada a dinâmica de todas squads trabalharem tanto com backend quanto com frontend, pois notamos que seria melhor que os integrantes pudessem se manter focados em uma tecnologia durante todo o projeto e seguimos esse pensamento o máximo possível.

Após a divisão das squads fui designado para a squad 3, que, em geral, estaria responsável da correção de bugs, porém acabei ficando responsável por desenvolver o Swagger\cite{swagger} do projeto por já ter experiência com a ferramenta. Consegui finalizar a tarefa com facilidade grande parte do que precisava ser feito. Além das funcionalidade visuais do Swagger\cite{swagger}, também queríamos adicionar a capacidade de testar os endpoints utilizando a \ac{ui}, para isso, foi necessário fazer um estudo acerca dessa utilidade.

Após finalizar minha task, resolvi auxiliar meu colegas com o redesenvolvimento de alguns endpoints para que apenas usuários que tivessem uma conta e estivessem autenticados conseguissem consumir algumas partes da \ac{api}.

Além de tarefas de desenvolvimento de código, estava constantemente auxiliando outros colegas da AGES com tasks, tanto de back, quanto de frontend, porém nessa sprint contribui majoritariamente com tarefas de backend, por estar altamente atrelado ao que eu havia desenvolvido na sprint.

Infelizmente, ao final da sprint, novamente, não conseguimos entregar tudo que foi planejado, porém alguns pontos que eram possíveis causadores disso foram observados e apontados, para que pudessem ser corrigidos o mais rápido possível. Acredito que a entrega da sprint 2 não tenha sido muito boa da perspectiva dos clientes, pois, apesar do backend da aplicação ter sido completamente entregue, a parte visual, ou seja, o frontend, estava com diversos débitos técnicos. 