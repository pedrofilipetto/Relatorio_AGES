\subsection{Sprint 4}

Após a curta duração da sprint 3 e a grande quantidade de débito técnico e correções para serem feitas, decidimos manter as mesmas squads para a quarta sprint, o que iria economizar tempo e permitir usá-lo para desenvolvimento. Por eu ser o integrante com mais experiência em desenvolvimento em Flutter\cite{flutter} e Dart\cite{dart}, as tarefas de criação de novas features foram designadas para minha squad, juntamente com algumas de correção de bugs.

Com isso, fiquei designado de desenvolver o QR Code que seria usado para atrelar um médico a um paciente. Essa task não se mostrou muito difícil de completar, dado que já havia trabalhado com modal na sprint passada e para a geração de um QR Code existem diversas \ac{api}s que poderiam ser utilizadas. Como consegui terminar minha tarefa rapidamente, procurei auxiliar o máximo de colegas possível sabendo que teríamos muito o que fazer nessa última sprint.

Comecei conversando com o colega responsável por desenvolver a tela para ler o QR Code para conectar um médico a um paciente, porém ele comunicou que acreditava estar fazendo um bom progresso sozinho, então fui procurar outras pessoas para auxiliar. 

Acabei auxiliando uma colega que estava responsável pela tela de cadastro. Essa tela havia sido desenvolvida na última sprint, porém apresentava diversos problemas, por isso, optei por indicar que seria uma boa ideia começar a tela do zero, por haver uma grande quantidade de problemas na estruturação do código, o que acarretaria outros problemas posteriormente. Essa correção levou um tempo para conseguir ser feita, dado que algumas outras utilidades da aplicação haviam sido alteradas, como, por exemplo, os validadores que estavam sendo usados nas áreas de escrita, por isso, tivemos que estudar o código e entender como estavam sendo usados e, assim, alterar a tela de acordo.

O último fim de semana antes da entrega foi extremamente corrido, pois muitas correções ainda não haviam sido finalizadas e descobrimos que a tela para leitura de QR Code não estava funcionando. Acabamos entrando em um grupo de aproximadamente 6 pessoas, incluindo pessoas de todos os níveis da \ac{ages}, para tentar corrigir o máximo de problemas possível. Ao fim conseguimos finalizar grande parte das correções essenciais, porém houve coisas que não conseguimos completar.

Embora todos decorridos listados tenham proporcionado grande dificuldade para a equipe, ao fim, os clientes se mostraram muito contentes com o resultado obtido. Apesar da satisfação dos clientes, houve problemas que não foram resolvidos a tempo da apresentação da sprint 4 aos clientes, porém, eles foram apontados e corrigidos dentro do prazo da apresentação final do projeto.