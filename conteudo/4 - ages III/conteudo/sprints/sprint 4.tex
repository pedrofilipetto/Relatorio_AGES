\subsection{Sprint 4}

No início da Sprint 4 foram, novamente, separadas as tasks para cada integrante da equipe, porém não teríamos quase nada de novas features, essa Sprint seria composta em sua maior parte por correção de bugs.

Durante essa Sprint eu fiquei responsável por corrigir alguns bugs que tínhamos encontrado com os audios e as imagens e por fazer uma integração com a \ac{api} Lumiar, uma \ac{api} da empresa das stakeholders. Essa \ac{api} deveria receber algumas informações dos usuários em alguns casos pré-determinados.

Como já haviamos recebido um documento listando todas as coisas que estavam erradas no projeto, resolvi começar arrumando os audios e imagens. Foi bem simples fazer as correções necessárias, dado que era basicamente mudar a nomenclatura no \ac{s3} e atualizar nosso seed do banco de dados.

A parte difícil dessa Sprint foi quando tive de fazer a integração. Inicialmente fiz uma chamada para a \ac{api} Lumiar toda vez que um usuário finalizava um módulo, porém, mais para o fim da Sprint recebemos um documento informando que essa integração deveria ser feita algumas vezes horas após o último acesso do usuário, mais especificamente 24 horas e 72 horas depois. Essa \ac{api} iria enviar uma notificação por WhatsApp para o usuário incentivando-o a voltar e terminar o módulo.

Como nunca havia feito isso, precisei pesquisar e pensar um pouco e cheguei a conclusão que usar cron expressions seria uma boa alterantiva. Pedi ajuda ao professor orientador perguntando se ele sabia de alguma tecnologia que pudesse ser usada de forma fácil e ele me reportou que o próprio Springboot possui essa utilidade. Com essas informações montei uma lógica que executava uma vez por dia e verificava quais usuários deveria ser enviados para a \ac{api} Lumiar.

Após a entrega as stakeholders se mostraram muito felizes com o resultado final do projeto, porém encontraram alguns erros no fluxo do guia montado por elas. Assim, pedimos, novamente, um reporte de todas alterações que seriam necessárias.

Notamos que diversas coisas que foram reportadas como erradas, foram, na verdade, alteradas durante a Sprint 4 para ficarem como pedido no documento enviando anteriormente, mas, refizemos algumas alterações para ficarem do gosto das stakeholders.