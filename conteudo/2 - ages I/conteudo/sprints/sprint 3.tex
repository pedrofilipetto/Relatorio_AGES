\subsection{Sprint 3}

No inicio da Sprint 3 foram divididas as squads e suas respectivas tarefas novamente, dessa vez, eu e minha squad ficamos incumbidos da criação do endpoint e dados mockados das telas e dicas do aplicativo.

Esta sprint ocorreu em uma época mais tranquila em relação à faculdade, possibilitando um maior empenho no projeto. Como teria mais tempo para trabalhar no projeto achei que seria uma boa ideia tentar algo que ainda não havia feito, por isso acabei decidindo fazer a rota para recuperar as dicas.

O trabalho feito no endpoint foi bem interesante e desafiador, já que era algo completamente diferente do que eu havia feito até o momento, mas foi possível completar a task sem grandes dificuldades.

Durante o andamento da sprint tivemos alguns problemas originados de algumas \ac{us} mal planejadas. Após as correções necesárias, tanto no código ja desenvolvido, quanto nas \ac{us}, as tarefas atribuídas a nossa squad foram finalizadas, restando ainda tempo para o fim da Sprint.

Por conta de termos finalizado nossa \ac{us} com antecedência, resolvi ir além e fazer os testes unitários do endpoint implementado e, também, organizar os dados mockados em diferentes arquivos json, mantendo assim, uma boa organização no projeto.

Embora tenha feito algo novo nessa sprint, tudo ocorreu de acordo com o esperado e sem problemas na entrega, tendo, novamente, uma cliente muito feliz com os resultados obtidos pelo time durante a terceira sprint. Felizmente, nessa Sprint conseguimos resolver todos os débitos técnicos repassados e, também, conseguimos finalizar todas tarefas planejadas, além de fazer um crédito técnico por ter realizado alguns dos testes unitários que seriam feitos em Sprints futuras.