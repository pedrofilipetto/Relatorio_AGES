\subsection{Sprint 0}

Durante a Sprint 0, o time focou no planejamento e na prototipagem das interfaces da plataforma Operações do \ac{gaeco}. Enquanto os colegas \ac{ages} II desenhavam o modelo inicial do banco de dados e os \ac{ages} III e IV definiam a arquitetura, minha contribuição inicial foi na colaboração com ideias para o desenvolvimento das telas.

Por ser minha primeira experiência na \ac{ages}, após a fase inicial de prototipagem, encontrei dificuldade em identificar os próximos passos e reconheço que faltou proatividade da minha parte para buscar novas tarefas. Decidi, então, focar em estudar as tecnologias para as futuras sprints de desenvolvimento do backend e frontend.

Para isso, me desafiei a praticar desenvolvendo telas do frontend, baseado nos mockups do \ac{figma}, e comecei alguns rascunhos procurando me aproximar ao máximo das telas desenhadas no \ac{figma}. Isso me fez entender um pouco da estrutura para que eu pudesse utilizar no desenvolvimento que viria em seguida.

A etapa da Sprint 0 foi concluída com a apresentação dos mockups e das User Stories da Sprint 1 aos stakeholders. Para mim, o momento mais significativo foi a retrospectiva, onde pude alinhar expectativas sobre comunicação e responsabilidades com os colegas mais experientes da equipe (\ac{ages} IV). Essa troca de ideias me tranquilizou e esclareceu, pois me fez perceber que a estrutura ser menos rígida e menor volume de tarefas são características próprias de uma sprint exploratória como a Sprint 0, diferentemente do que ocorre nas fases seguintes de desenvolvimento.

A lição mais importante desta sprint foi a importância da comunicação e da proatividade. Aprendi que é minha responsabilidade buscar ativamente o auxílio da equipe e pedir direcionamento. Ao final da Sprint 0, me senti tecnicamente mais preparado e esclarecido para executar as tarefas de frontend e, principalmente, mais ciente da postura colaborativa que o projeto exige para as próximas etapas.
