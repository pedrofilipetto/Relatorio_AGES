\subsection{Sprint 4}

Novamente, como nas sprints passadas, no início da Sprint 4 foram divididas 
as tasks e as squads, dessa vez fui atribuído a squad Água, que estava responsável 
pela seleção de uma \ac{api} de notícias, integração da \ac{api} com 
o programa, criação do carrossel de notícias e do card de notícias.

Nessa última sprint retornei para o desenvolvimento do front end do projeto, 
pois foi a área que mais me interessou e que eu mais estudei. Após esta decisão fiquei 
responsável pela criação do card de notícias, que deveria seguir o padrão feito no 
\cite{figma}. 

A produção do card não foi algo que demonstrou muita dificuldade, porém eu 
ainda não havia trabalhado com imagens, portanto precisei pesquisar e descobrir 
como faria para carregar uma imagem a partir de uma \ac{url}, assim como aprender a aplicar um gradiente de opacidade ao longo da imagem.

Quando terminei o card notei que a pipeline estava dando erro, porém não 
parecia ser um problema vindo do meu código, por isso, fui procurar a fonte do 
erro e descobri que ela estava excedendo o tempo máximo de execução do script. 
Informei aos colegas que conseguiriam lidar com o problema e aguardei a correção.

Após a correção da pipeline, quando juntamos tudo que havia sido feito, 
notamos que a \ac{api} que havíamos escolhido tinha um limite de 5 chamadas por dia, 
assim, foi necessário procurar outra \ac{api} que nos entregasse o que precisávamos. Por 
fim, foi necessário mockar os dados, já que estávamos nos aproximando da data de 
entrega e não havíamos encontrado uma \ac{api} que entregasse o que precisávamos. 

Ao fim da Sprint, embora tenhamos encontrado alguns problemas, tudo foi 
entregue. Tivemos um feedback muito positivo vindo da cliente, que estava muito 
contente com os resultados que obtivemos ao longo do projeto. 