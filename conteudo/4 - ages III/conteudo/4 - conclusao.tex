\section[Conclusão]{Conclusão}

Em suma, com a junção das experiências obtidas como \ac{ages} I, II e III, posso afirmar que cada time tem uma dinâmica com relação à divisão de squads e organização de tasks. Diferentemente das minhas duas experiências anteriores, nessa \ac{ages} não foram feitas squads, mas sim divisão de tasks diretamente para cada integrante da equipe. Acredito fortemente que essa organização não deve ser comumente usada e depende fortemente da proatividade dos colegas do time.

Com relação à minha atuação como \ac{ages} III, posso concluir que, apesar de termos muitos \ac{ages} III, consegui aplicar meu conhecimento técnico de forma positiva para o projeto, auxiliando os colegas que precisavam e assumindo o controle quando necessário. Apesar de não ter tido muito contato com o frontend da aplicação sabia constantemente o que estava sendo desenvolvido tanto no front quanto no backend e estava observando e avaliando o que poderia dar errado.

No quesito social, ou seja, softskill, eu acredito ter crescido muito desde minha \ac{ages} II, pois consegui ter um melhor contato com os colegas e repassar de uma maneira mais didática o meu conhecimento técnico, mostrando uma clara evolução dessa habilidade.