\subsection{Sprint 1}

Com o planejamento concluído, a Sprint 1 marcou o início efetivo do desenvolvimento da plataforma. Fui responsável por duas tarefas principais que abrangeram tanto o backend quanto o frontend, permitindo-me aplicar os conhecimentos adquiridos na fase de estudos. No backend, implementei o endpoint para listagem de casos (`GET /cases`), que incluía a lógica de paginação, filtragem dinâmica e a proteção de rota via token de autenticação. Este trabalho exigiu colaboração com o colega Bryan Leandro, para integrar a funcionalidade com o sistema de login, e foi refinado com base no feedback dos colegas Felipe Cardona e Gabriel Belmonte.

No frontend, desenvolvi um componente de `Input` genérico e reutilizável com React e Material-UI \cite{materialui}, seguindo as especificações do Figma, os princípios de componentes do NextJs e alinhando as decisões de implementação com as colega Lara Kunrath e Carolina Ferreira. Além das minhas tarefas designadas, procurei manter uma postura colaborativa, auxiliando os colegas Rodrigo Schmitt e Gabriel Pinho com suas tarefas.

O principal desafio desta sprint não foi técnico, mas sim de processo e comunicação dentro da equipe. Observei que a falta de uma sincronia técnica inicial entre desenvolvedores trabalhando em partes interconectadas do sistema levou a inconsistências no padrão de código, o que gerou retrabalho no final da sprint. Essa experiência reforçou a lição da sprint anterior sobre a importância da proatividade.

Aprendi que, para garantir a consistência e a qualidade da base de código, a comunicação técnica no início do desenvolvimento de uma tarefa é tão crucial quanto o code review no final. Ao final da Sprint 1, ambas as minhas tarefas foram concluídas e integradas com sucesso. Sinto que solidifiquei meu conhecimento prático em FastAPI e NextJs e, mais importante, obtive uma visão mais clara de como minha comunicação proativa pode contribuir diretamente para a qualidade técnica e a eficiência de toda a equipe.