\section[Desenvolvimento do Projeto]{Desenvolvimento do Projeto}

\subsection{Repositório do Código Fonte do Projeto} % TODO: reescrever
  Para este projeto foram utilizados dois repositórios separados, ambos foram mantidos no GitLab da \ac{ages}. A separação deles foi feita em frontend, que foi feita com a tecnologia Next.Js\cite{nextjs}, usando a linguagem TypeScript\cite{typescript} e em backend que foi feita com a tecnologia FastAPI\cite{fastapi}, utilizando a linguagem Python\cite{python}.
  
    \begin{itemize}
      \item Vincula frontend: https://tools.ages.pucrs.br/vincula/frontend
      \item Vincula backend: https://tools.ages.pucrs.br/vincula/backend
    \end{itemize}

\subsection{Banco de Dados Utilizado}
  Para o desenvolvimento do projeto, os colegas AGES II escolheram o banco de dados relacional PostgreSQL\cite{postgresql}. A escolha foi feita por se tratar de um banco de dados versátil e escalável, atendendo os requisitos do projeto.

  % \begin{figure}
    % TODO: MODELO BANCO
  % \end{figure}

    
\subsection{Arquitetura Utilizada} % TODO: reescrever
  A arquitetura utilizada no projeto é baseada no modelo cliente-servidor, que consiste em um banco de dados, um servidor e clientes que se conectam ao servidor para obter acesso às informações desejadas.

\subsection{Protótipos das Telas Desenvolvidas} % TODO: reescrever
  A prototipagem das telas foi feita após a definição do design do produto e das funcionalidades, que foram definidas juntamente com a stakeholder. Os protótipos foram produzidos utilizando a plataforma Figma\cite{figma}.

    %\begin{itemize}
      % \item TODO: LINK WIKI MOCKUPS
    %\end{itemize}

\subsection{Tecnologias Utilizadas}
  O projeto foi construido utilizando três tecnologias principais, sendo elas o React\cite{react} com NextJs\cite{nextjs} e a linguagem TypeScript\cite{typescript} para o frontend, o FastAPI\cite{fastapi}, com a linguagem Python\cite{python} para o backend e PostgreSQL\cite{postgresql} para o banco de dados.
  
  O NextJs\cite{nextjs} foi escolhido para o frontend por ser uma framework moderna e que abstrai diversos aspectos trabalhosos do desenvolvimento com a biblioteca React\cite{react} pura.

  No backend, logo de início, decidimos que uma framework Python\cite{python} seria essencial para o projeto, visto a necessidade de manipular grandes volumes de dados de forma eficiênte. Por isso, optamos pela FastAPI\cite{fastapi}, que torna o processo de programar endpoints simplificado.