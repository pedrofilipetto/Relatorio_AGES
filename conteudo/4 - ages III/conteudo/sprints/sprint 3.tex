\subsection{Sprint 3}
Com o início dessa Sprint foram, novamente, divididas as tasks que cada integrante da equipe seria responsável por desenvolver. Aproveitei para avisar o time de que grande parte dessa Sprint eu estaria ausente por conta de uma viagem proveniente de uma competição que eu e alguns amigos ganhamos com a \ac{uci} e, por isso, estaria nos Estados Unidos durante a próxima semana o que poderia dificultar no desenvolvimento da minha task.

Durante essa Sprint eu, juntamente com os outros \ac{ages} III, recebemos as tasks de mapear e organizar o items dos módulos a partir do documento apresentado pelas stakeholders. Eu imaginei que não conseguiria finalizar minha task até o dia da viagem, por isso, acabei conversando com um outro \ac{ages} III para que ele pudesse ver alguém para assumir a parte da minha task que faltava.

Além dessa task, eu recebi um outro trabalho para fazer, criar um \ac{s3} e adicionar nele os áudios que deveriam ser apresentados no frontend. Acabei preferindo focar em finalizar essa task, já que eu era o único com experiência em \ac{aws} no time. Assim, enviei uma requisição para o arquiteto da \ac{ages} pedindo que o mesmo criasse um \ac{s3} para que pudessemos utilizar. Enquanto aguardava fui tentando mapear quais áudios deveriam ir para o \ac{s3} e em qual ordem seriam utilizados, para que pudesse manter o armazenamento de forma organizada e que permitisse acessos eficientes. Quando recebi acesso ao \ac{s3} apenas precisei fazer o upload dos arquivos e definir os meta-dados.

No fim acabei precisando repassar minha task de mapear os items dos módulos para outro colega, dado que minha viagem iria percorrer até o sábado anterior à entrega.

Quando retornei de viagem, já sabia que seria necessário fazer o deploy das alterações para a \ac{aws}, porém não esperava encontrar um problema com o espaço de armazenamento da máquina enquanto fazia isso. Eu, juntamente com outro \ac{ages} III ficamos por algumas horas tentando resolver o problema até descobrirmos que isso estava ocorrendo por conta do docker, que não estava fazendo a remoção dos caches de building, mantendo todos os arquivos que utilizávamos para montar qualquer container duplicado dentro da máquina. Apesar dos problemas conseguimos fazer o deploy do front e backend.

Ao fim da Sprint 3 fizemos, no geral, um bom trabalho, não tivemos débitos técnicos, porém, diversos erros foram encontrados, o que resultaria em uma quantidade de trabalho bem maior que o esperado para a próxima e última Sprint. Apesar dos erros, as stakeholders pareceram bem felizes com a entrega.