\section[Conclusão]{Conclusão}

Em suma, com a junção das experiências obtidas como \ac{ages} I e II, posso afirmar, que nem sempre ter uma grande quantidade de integrantes em uma equipe irá aumentar o rendimento dela, dado que existirá mais pessoas para se comunicar, aumentando a complexidade e dificuldade da organização e divisão de tasks e controle do que está sendo feito no momento. Ou seja, embora uma equipe muito pequena não seja tão boa pela falta de “mão de obra”, uma equipe muito grande terá uma maior dificuldade em organização e comunicação, ambas tendo seus pontos positivos e negativos.

Também pude observar que, mesmo que alguns integrantes da equipe se mostrem extremamente interessados e proativos no desenvolvimento do projeto, caso outros não tenham tanto empenho ou interesse o projeto será afetado, pois, como um professor falou em uma aula, “a eficiência de uma equipe é limitada pelo seu elo mais fraco”, ou seja, de nada adianta que todos consigam terminar suas tarefas de forma rápida se um integrante estiver trancado em uma task. Para resolver isso é necessário constante observação e auxílio no que está sendo feito pelos outros integrantes da equipe, para que ninguém fique trancado no desenvolvimento de suas tarefas.

Com relação a minha atuação como \ac{ages} II, posso concluir que pude aplicar o conhecimento obtido como \ac{ages} I, tanto em soft skills como em hard skills. No caso das soft skills, comecei muito mais cedo a me comunicar com a equipe e identificar problemas e dificuldades que cada um estava tendo. Já em hard skills, como havia trabalhado com Flutter como AGES I, era um dos únicos integrantes que possuía experiência com a ferramenta, por isso, acabei sendo de grande auxílio a todos, ajudando e ensinando sempre que necessário.