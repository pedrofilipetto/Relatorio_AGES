\subsection{Sprint 1}

Com o planejamento concluído, iniciamos com tarefas para o desenvolvimento das telas em duplas. Até o meio da Sprint 1, desenvolvemos um esboço para a tela de listagem de alvos que foi atribuída a mim e ao colega Felipe Cruz, porém, em seguida os colegas \ac{ages} IV julgaram melhor que apenas deveríamos desenvolver os componentes das telas para que após esse processo pudessemos apenas uni-los. 

Após replanejarmos a Sprint 1, iniciamos efetivamente o desenvolvimento da aplicação. Fui responsável por dois componentes que abrangeram o frontend, permitindo-me aplicar os conhecimentos adquiridos na fase de estudos. No frontend, implementei a search bar para que tornasse possível a busca por algo nas telas, o segundo componente foi o team indicator member que tornou possível vizualizar os agentes adicionados nas operações. Este trabalho exigiu colaboração com o colega Felipe Cruz, e foi feita a validação dos componentes e integração no projeto pelos colegas Erick Muniz e Marcus Raach.

Além das minhas tarefas designadas, procurei sempre manter uma postura colaborativa, auxiliando os colegas com suas tarefas e continuei estudando para as tarefas seguintes de backend. 

O principal desafio desta sprint foi o replanejamento em meio a Sprint, o que na minha opinião desacelerou a Sprint, mas acredito ser uma decisão crucial para a eficiência dos processos. Observei uma falta de convicção entre gerentes de projeto \ac{ages} IV para a tomada de decisão inicial. Essa experiência reforçou a lição para melhorarmos a comunicação do time.

Aprendi que, para garantir a consistência e a qualidade da base de código, a comunicação técnica no início do desenvolvimento de uma tarefa é tão crucial quanto o code review no final. Ao final da Sprint 1, ambas as minhas tarefas foram concluídas e integradas com sucesso. Sinto que solidifiquei meu conhecimento prático em frontend e React Native e, mais importante, obtive uma visão mais clara de como a comunicação e proatividade pode contribuir diretamente para a qualidade técnica e a eficiência de toda a equipe.