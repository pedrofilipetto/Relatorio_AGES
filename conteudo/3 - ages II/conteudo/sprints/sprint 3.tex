\subsection{Sprint 3}

Com o início da terceira sprint, o time foi separado em equipes de 5 pessoas, novamente, buscamos manter a forma de organizar as squads focadas em áreas específicas, pois essa técnica mostrou um melhor resultado do que as squads terem responsabilidades tanto no backend, quanto no frontend.

Nessa sprint voltei a ser um membro da squad 2 e ficamos responsáveis por desenvolver algumas telas e componentes do fluxo dos médicos. A parte definida para mim inicialmente foi desenvolver a tela de menu do médico, porém notamos que essa sprint teria uma duração bem reduzida, de cerca de 2 semanas, por isso, acabei reorganizando as tasks e peguei tarefas com uma alta prioridade por ter mais experiência e tempo livre para realizá-las no período determinado. 

Com isso, acabei ficando responsável por desenvolver o componente de perfil do médico, no qual seria apresentado sua imagem, seu nome, endereço do consultório, especialização e descrição. Comecei a trabalhar o mais cedo possível para que entregasse dentro do prazo esperado, porém a task se mostrou mais difícil de finalizar do que o esperado, pois muitas outras tarefas menores estavam “escondidas” dentro dela, como por exemplo a criação de uma modal para selecionar os convênios aceitos pelo médico e as variações de visualização e de edição do componente e, por isso, o prazo acabou sendo excedido em um dia.

Ao finalizar a criação do componente me mostrei disponível para auxiliar os colegas em suas respectivas tasks, auxiliando, também, colegas de outras squads que estavam com suas tarefas trancadas e não sabiam como avançar.

Apesar de nossos esforços para terminar o que foi combinado no período da terceira sprint, não conseguimos entregar grande parte do que era esperado, além disso, o que foi entregue possuía vários problemas a serem corrigidos, por isso, a próxima e última sprint teria que ser muito bem planejada e organizada para que o projeto seja finalizado e entregue de forma completa e satisfatória para os cliente.