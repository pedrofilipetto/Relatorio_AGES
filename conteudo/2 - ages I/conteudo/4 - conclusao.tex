\section[Conclusão]{Conclusão}

Em suma, até o momento, posso dizer com certeza que a AGES foi a melhor experiência que tive dentro da faculdade, pois ela possibilita uma simulação de um ambiente real de trabalho dentro da faculdade, proporcionando um aprendizado que não é possivel obter dentro das salas de aula normais.

Com relação a minha participação nas Sprints 0 e 1, concluo que é necessária uma melhora de comunicação com o time, já que contatei poucos companheiros fora da minha squad. Porém com relação a participação da parte técnica do projeto, creio ter atuado de maneira muito satisfatória, já que absorvi muito conhecimento e consegui aplicar o mesmo no projeto.

Já, nas Sprints seguintes consegui melhorar minha comunicação com o resto do time, conseguindo até auxiliar alguns colegas e informar problemas ocorridos. Com relação ao desempenho técnico acredito que tenha melhorado, já que acabei participando de partes do projeto que não havia tido oportunidade ainda.

Aprendi, também, que, embora minha squad tenha conseguido entregar tudo que foi prometido, é necessário entrar em contato com os outros integrates da equipe para verificar se eles necessitam de ajuda com algo, já que somos um time, e o erro de um integrante é um de todo o grupo.

Para finalizar, concluo que a experiência ganha durante a AGES I foi de imensa importância para o meu desenvolvimento como Engenheiro de Software. Aprendi que a comunicação com a equipe é algo de extrema importância e que uma comunicação ineficiente, ou até mesmo a falta dela, pode acarretar em sérios problemas no projeto.