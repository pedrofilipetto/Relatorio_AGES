\subsection{Sprint 2}

No início da Sprint 2 foi debatido se iríamos continuar a trabalhar sem squads e apenas delegando tasks e resolvemos continuar com o sistema de delegar tasks, monitorar o progresso dos colegas e, caso alguém tivesse dúvidas ou precisasse de ajuda com o desenvolvimento poderia pedir ajuda no Discord ou WhatsApp. Essa decisão foi feita por termos tido um ótimo resultado na Sprint 1 e optamos por não atrapalhar a dinâmica que estava se formando entre os integrantes da equipe.

Enquanto era feita a divisão de tasks, foi alertado que essa seria a Sprint mais díficil e que caso as tasks dela não fossem bem implementadas o resto do projeto iria ser muito mais complicado do que o esperado, o que adicionou um certo peso e seriedade sobre o trabalho que seriad desenvolvido. Isso se dava ao fato da nossa aplicação ser relativamente modular, o que implicava na necessidade de uma ferramenta que fizesse uma boa organização do que era recebido da nossa \ac{api} e, também, em uma boa organização na forma de guardar os dados que seriam expostos no frontend, isso exigiria um trabalho com constante comunicação entre os desenvolvedores do backend e do frontend.

Durante essa Sprint dei procedimento à criação dos nossos recursos na AWS. Apesar de ter feito o envio com o Terraform completo do que seria usado, ocorreram problemas com a instanciação dos recursos utilizando a ferramenta e acabei pedindo para o arquiteto da AGES criar os recursos manualmente para mim. Quando recebi os acessos, a primeira coisa que fiz foi inicializar nosso backend na máquina EC2 utilizando Docker e realizar os testes necessários. Após isso tentei fazer o setup do Gitlab Runner também utilizando Docker, porém, após seguir diversos tutoriais que encontrei, nenhum funcionou. Por isso decidi fazer a instalação direto na máquina e tomei todas precauções possíveis para minimizar o entrelaçamento dos ambientes. Quando estava terminando de criar o Runner do frontend o Gitlab da AGES ficou fora do ar, o que impossibilitou a progressão por mais 3 dias.

Ao fim da Sprint consegui finalizar o que era necessário e coloquei a aplicação no ar, o que facilitou muito a apresentação para as clientes, dado que as mesmas não poderiam comparecer presencialmente na reunião. Com isso, pudemos receber um feedback detalhado do que precisaria ser corrigido e o que estava bom. Apesar da instabilidade no Gitlab conseguimos realizar uma entrega satisfatória para as clientes que se mostraram extremamente felizes com a velocidade e qualidade com a qual a aplicação estava sendo desenvolvida.