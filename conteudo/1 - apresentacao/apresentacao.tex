\chapter[APRESENTAÇÃO DA TRAJETÓRIA DO ALUNO]{APRESENTAÇÃO DA TRAJETÓRIA DO ALUNO}

Desde que me lembro, sempre fui ligado com computadores, explorando diversas áreas como jogos, edição e hardware. A programação, embora nunca
tenha sido um foco exclusivo, também fez parte desse interesse inicial, com um curso sobre tecnologia HTML.

Comecei minha vida profissional em outras áreas, como assistente de financeiro e suporte ao cliente e deixei os estudos de lado, porém na empresa onde trabalho hoje tenho perspectiva de crescimento principalmente na área de programação. Por isso, entrei na faculdade com um receio e com a missão de descobrir se seria realmente o que eu seguiria fazendo. Confesso que, antes de iniciar o curso, eu tinha a ideia de que não gostaria de atuar como programador. Hoje em dia, não me vejo escolhendo outra profissão que não seja na área da programação, conforme o tempo passa adquiro conhecimento, planejo os próximos passos e vou descobrindo qual função nessa área pretendo desempenhar no futuro.

Atualmente, como aluno da disciplina de Prática na AGES I, vivo experiências que se aproximam muito do que vejo na empresa que trabalho, o que é fundamental para solidificar minha visão de carreira e entender a realidade da profissão no mercado de trabalho. Elas me mostram que a criação de um software de sucesso vai muito além do código e envolve estratégia, planejamento e foco na experiência do usuário. Meu objetivo nesta fase do curso é aprimorar minhas habilidades técnicas de programação e, ao mesmo tempo, buscar ativamente oportunidades para aprender mais sobre todas as áreas.  