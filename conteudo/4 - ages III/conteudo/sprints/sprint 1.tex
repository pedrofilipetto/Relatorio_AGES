\subsection{Sprint 1}

No início da Sprint 1 tivemos que decidir entre trabalhar em squads ou apenas delegar tasks e quem necessitasse de ajuda com sua task iria atrás dos AGES III e IV para pedir auxilio. Acabamos indo com a segunda opção por causa da grande quantidade de AGES III na nossa equipe, o que iria dificultar a separação de squads.

Enquanto delegávamos as tasks notamos que nosso projeto era relativamente simples programaticamente e que não seria necessário atribuir tasks aos AGES III, permitindo que nós agissimos como "coringas" que poderiam ajudar em qualquer momento e também iríamos supervisionar tasks mais complicadas, além de fazer todos os reviews e correções necessárias no código.

Com relação às regras de envio de código, definimos que dois reviews seriam necessários para aprovar as mudanças. Um review de um AGES III com especialidade na área e um outro review dos AGES IV para termos certeza que nada passou despercebido. Isso funcionou muito bem, dado que não tivemos nenhum código com problemas enviado para nossa branch principal.

Durante o andamento eu fiquei incubido com o objetivo de modelar nossa estrutura cloud dado que a pessoa com maior conhecimento nessa área e experiência com AWS. Para isso defini uma estrutura que facilitasse ao máximo os nossos deploys, evitando que ocorressem problemas de última hora e que o máximo de etapas pudessem ser automatizadas. Assim montei um sistema que utilizaria o AWS Amplify para fazer o hosting do nosso frontend, permitindo um fácil build e deploy do nosso projeto, porém, não notei que o recurso não tinha suporte direto para ambientes self-hosted, o que dificultaria um pouco o processo de automação de deploy. Para o backend optei por utilizar uma EC2 que iria executar tanto nosso banco de dados quanto a nossa API. Essa EC2 estaria conectada a um API Gateway que iria ser a porta de entrada para nossa API, redirecionando todas chamadas para a máquina virtual. Para fazer o instanciamento de todos esses recursos na AWS criei uma configuração utilizando Terraform e enviei para o Arquiteto de Software da AGES para que pudesse ser aprovado.

Durante a Sprint 1 ocorreram pedidos de mudança nas especificações, o que significou que, infelizmente, alguns trabalhar que já haviam sido feitos deveriam ser alterados, porém, apesar de ser um acontecimento chato, não apresentou um grande problema dado que ainda estávamos no inicio do projeto. Também tivemos alguns problemas de comunicação entre alguns membros da equipe, o que resultou em algumas complicações na divisão de tasks, porém tudo ocorreu bem e conseguimos entregar mais tasks do que foram planejadas para a Sprint 1, se concluindo com stakeholders felizes e satisfeitas.