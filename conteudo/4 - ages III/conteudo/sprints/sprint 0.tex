\subsection{Sprint 0}

Na Sprint 0 foi feito o primeiro contato com uma das nossas stakeholders, conhecemos da origem da plataforma, como a ideia surgiu e como ela funciona. Também ficamos sabendo de que a plataforma já possuia uma estrutura definida e que isso não poderia ser alterado, dado que se tratava de um processo definido e criado por uma psicóloga. Também ficamos sabendo que existia um documento com diversas definições de design e recebemos a informação que seria necessário criar uma plataforma com o menor custo possível, dado que seria algo a ser disponibilizado para a sociadade de forma gratuita.

A partir disso começamos a projetar o design da plataforma no figma, porém, diferentemente das minhas experiências passadas na AGES, dessa vez não possuíamos um integrante com uma grande afinidade com o Figma e conhecimento de design, o que dificultou consideravelmente as atividades previstas para a Sprint 0. Por esse motivo, resolvemos iniciar trabalhando individualmente e criar diversos designs diferentes para que no próximo encontro fosse possível definir quais nós mais gostamos e seguir em diante.

Conseguimos facilmente chegar uma definição de design para a plataforma no formato mobile, porém estavamos com bastante dificuldade para o desktop, porém com o empenho de alguns integrantes conseguimos chegar a um resultado satisfatório para mostrar às stakeholders, mas algumas alterações ainda seriam necessárias e isso foi passado para a Sprint 1 como um débito.

Além do design, utilizamos a Sprint 0 para fazer a definição de quais técnologias seriam utilizadas no projeto a partir de votação em quais técnologias eram mais conhecidas e quais despertavam o interesse dos integrantes da equipe. Ao final da votação acabamos ficando com Typescript com React e Tailwind CSS para o frontend e Java com Springboot para o backend, assim como PostgreSQL para o banco de dados.

Após definir as técnologias, nós, AGES III começamos a projetar a base do projeto, inicializando o código e definindo dependências, além de fornecer exemplos para que os AGES I e II pudessem seguir quando fossem realizar suas tasks.