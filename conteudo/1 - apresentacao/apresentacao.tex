\chapter[APRESENTAÇÃO DA TRAJETÓRIA DO ALUNO]{APRESENTAÇÃO DA TRAJETÓRIA DO ALUNO}

Neste tópico o aluno deve fazer um paralelo da evolução dele ao longo do curso e sua trajetória pessoal prática (estágios, bolsas de pesquisa, voluntariado, hackatonas, emprego).
  
Ages I,II e III podem apresentar seu momento atual.

Não se esqueça de utilizar acrônimos para Siglas, como no exemplo: \ac{ages}.

Utilize referências também, modificando o arquivo \textit{bibliografia/bibliografia.bib} e citando-os com o comando\cite{artigo}.

Iniciei o curso de Bacharelado em Engenharia de Software enquanto a pandemia de Covid-19 ainda estava ocorrendo, no segundo semestre de 2021. Cursei, também, Bacharelado em Direito na PUCRS, porém acabei trocando para Engenharia de Software depois de três semestres, pois notei que o curso não se encaixava no meu perfil.

No segundo semestre do curso, eu e mais 5 outros colegas e amigos participamos da 5ª Hackatona de Engenharia de Software da PUCRS, o que me proporcionou uma experiência única, me permitindo aprender a trabalhar em grupo de um forma que ainda não havia tido a oportunidade.

No terceiro semestre comecei a AGES I e, juntamente, comecei a trabalhar como monitor de Programação Orientada a Objetos. A monitoria me proporcionou um aprendizado imenso acerca de como repassar meu conhecimento para os que estão precisando e a AGES está me compelindo a buscar mais conhecimento por conta própria.

Ao fim do terceiro semestre deixei de ser monitor e comecei a trabalhar como estagiário no \ac{lis} o que me proporcionou um grande conhecimento acerca de \ac{devops} e microsserviços, principalmente envolvendo AWS.

No quarto semestre, que ocorreu em 2023/1, eu atuei como AGES II no projeto Sow Good, no qual foi desenvolvida uma aplicação para auxiliar e melhorar a comunicação e transferência de informação entre pediatras e os pais de pacientes. Por ter aprendido muito sobre backend, banco de dados e microsserviços no estágio, consegui auxiliar bastante meus colegas no projeto, tanto na decisão do banco de dados e no que seria usado da AWS, quanto em quais tecnologias iríamos utilizar para desenvolver o projeto. Além disso, durante uma cadeira eu e mais dois amigos nos juntamos a um time americano participamos de uma competição da \ac{uci}, a Beall and Butterworth Competition - International Track. Ao fim do semestre recebemos a notícia que a nossa equipe havia vencido e que a premiação seria uma viagem para conhecer a universidade e presenciar a competição.

Durante os próximos dois semestres minha trajetória acadêmica e profissional se manteve estável, até o final do semestre de 2024/1, quando decidi sair do estágio no \ac{lis} para abrir uma Startup com um colega. A Startup se chamava Beasybox e tinha como objetivo criar automações para os processos de consultoria empresarial. Somando a extensa experiência do meu colega em consultoria empresarial com a minha experiência em programação e infraestrutura, acreditávamos que conseguiriamos fazer uma solução funcional e elegante. Conseguimos entrar no TecnoPUC e criamos parcerias com outras Startups presentes no ecossistema, o que nos ajudou a impulsionar a jornada inicial da empresa.

Apesar dos nossos planos, pouco tempo após fundar a Startup tivemos o maior desastre natural do último século e, juntamento com nosso investidor resolvemos juntar esforços, de forma voluntária, com outros colegas do curso de Engenharia de Software para desenvolver uma plataforma e auxiliar empresas que foram atingidas pela enchente.

Cerca de 2 meses após abrir a Beasybox, em 2024/2, eu fui para uma mobilidade acadêmica em Lisboa, Portugal. Essa experiência me proporcinou momentos incríveis de crescimento acadêmico, profissional, cultural e pessoal. Lá eu consegui uma bolsa de trabalho dentro da Universidade, na qual gerenciei as salas de visualização avançada e de aulas híbridas. Além disso, desenvolvi um novo sistema de autenticação e identificação de estudantes.

No semestre de 2025/1 eu voltei para o Brasil, porém segui com a bolsa até o final de Abril. Também comecei a atuar como AGES III no projeto Treinamento Autoguiado, que era uma aplicação para auxiliar pessoas a melhorar seu autocontrole e sua maneira de lidar com o estresse.
TODO