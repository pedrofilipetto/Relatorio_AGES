\subsection{Sprint 1}

No começo da Sprint 1 foram divididas as squads, para que todos pudessem trabalhar e experimentar todas as tecnologias utilizadas no projeto, todas squads possuíam tasks, tanto de backend, quanto de frontend. Após a divisão dos integrantes da equipe entre as squads eu fui designado para a squad 2, que tinha a responsabilidade de desenvolver o cadastro do paciente.

Dentro da minha squad eu fiquei responsável por desenvolver a tela para adicionar os responsáveis do paciente e de criar o endpoint da \ac{api} que iria cadastrar a conta no 
Firebase Authentication\cite{firebaseauth} e salvar os dados do paciente no banco de dados.

Por já possuir uma experiência prévia com Flutter\cite{flutter} consegui desenvolver a tela sem muitos problemas. Durante o desenvolvimento da tela notei que seria melhor a criação de um componente especial para aceitar a escrita do usuário já que o componente existente estava recebendo muitas variáveis, tornando-o muito complexo para o entendimento.

Para o desenvolvimento do endpoint de cadastro foi necessário um pouco de estudo, já que nunca havia utilizado ExpressJs\cite{expressjs} para lidar com as rotas. Nessa parte tive um pouco mais de dificuldade, pois não possuía tanta experiência quanto em Flutter\cite{flutter}, porém consegui desenvolver um endpoint funcional. Ao longo da Sprint 1 ela precisou ser alterada para atender algumas alterações que estavam sendo realizadas na arquitetura do projeto, porém foi finalizada a tempo para a entrega. 

Além de realizar minhas tarefas, ajudei, também, vários \ac{ages} I com suas respectivas tasks, ensinando qual nomenclatura deveria ser utilizada, em quais pastas cada componente e tela deveria estar, e, também ajudando a desenvolver e finalizar as tasks. 

Ao fim dessa sprint não conseguimos entregar tudo que estava planejado para os clientes, ficando assim, com alguns débitos técnicos para a próxima sprint, porém, apesar dos problemas, eles gostaram bastante do que foi feito. Acredito fortemente que esse débito técnico ocorreu por ainda estarmos nos conhecendo como time o que diminui a produtividade da equipe como um todo.