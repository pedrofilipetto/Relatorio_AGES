\subsection{Sprint 0}

Durante a Sprint 0, o time focou no planejamento e na prototipagem das interfaces da plataforma Vincula. Enquanto os colegas AGES II desenhavam o modelo inicial do banco de dados e os AGES III e IV definiam a arquitetura, minha contribuição inicial foi na colaboração com os mockups das telas essenciais no Figma\cite{figma}, como as de Home, Login e a de Visualização de Casos.

Por ser minha primeira experiência na AGES, após a fase inicial de prototipagem, encontrei um desafio em identificar os próximos passos e reconheço que faltou proatividade da minha parte para buscar novas tarefas. Vendo isso como uma oportunidade, decidi focar em me capacitar tecnicamente para as futuras sprints de desenvolvimento do backend.

Para isso, desenvolvi um projeto de estudo prático com FastAPI\cite{fastapi}, implementando funcionalidades de cadastro de usuários, login e autenticação com JWT, o que me deu uma base sólida para as tarefas que viriam a seguir. Adicionalmente, para entender a complexidade dos dados que iremos manipular, criei scripts com a biblioteca Pandas\cite{pandas} para fazer uma análise exploratória nos arquivos de exemplo do SIMBA e SITTEL, o que me permitiu entender na prática como os vínculos se formam.

A sprint terminou com a apresentação dos mockups e das User Stories planejadas para a Sprint 1 aos stakeholders. Posteriormente, a retrospectiva da Sprint 0 foi uma experiência de grande importância para meu aprendizado, pois nela pude esclarecer minhas dúvidas e alinhar com os colegas AGES IV questões sobre a comunicação e responsabilidades. A conversa me acalmou quanto às incertezas sobre a organização de tarefas, pois entendi que a natureza da Sprint 0 é realmente mais fluida e exploratória, sem uma definição de atividades individuais tão rígida como nas sprints de desenvolvimento.

A lição mais importante desta sprint foi a importância da comunicação e da proatividade. Aprendi que é minha responsabilidade buscar ativamente o alinhamento com a equipe e pedir direcionamento. Ao final da Sprint 0, sinti-me tecnicamente mais preparado para as tarefas de backend e, principalmente, mais ciente da postura colaborativa que o projeto exige para as próximas etapas.