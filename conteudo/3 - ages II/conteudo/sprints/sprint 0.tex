\subsection{Sprint 0}

Durante o andamento da Sprint 0, foi definido, junto com os stakeholders, um 
padrão para ser seguido no design do projeto, porém, como foi descrito anteriormente, 
a aplicação ainda não possuía uma identidade visual bem definida, por isso, alguns 
parâmetros foram estabelecidos com eles, como, por exemplo, evitar a utilização de tons pasteis, que são cores com tons mais pálidos. Também foi definido que seria 
melhorar utilizar componentes mais arredondados, para serem mais atrativos e amigáveis para os possíveis usuários.

Nessa etapa não vimos necessidade de fazer divisão de tasks por squads, por isso, combinamos o que cada um iria fazer por afinidade com a ferramenta Figma\cite{figma}. Foi delegado para mim a atividade de projetar a tela de apresentar o QR Code do médico, que seria a maneira como os pacientes iriam conectar a sua conta com a do médico. Acabei desenvolvendo diversos protótipos até chegar em um que achei que encaixaria bem com a identidade visual que estava se formando.

Além do desenvolvimento dessa tela, também auxiliei outros colegas no desenvolvimento do 
protótipo deles, fiz a revisão para que todas as telas estivessem usando os 
componentes desenvolvidos no style guide e me certifiquei que a navegação dos 
mockups estava correta e funcional para a apresentação para os stakeholders.

Além de desenvolver os mockups, eu criei um esquema conceitual para o 
banco de dados. Para isso utilizei a ferramenta LucidChart\cite{lucidchart}, que permite a criação de diagramas, fluxos e até mesmo mockups de forma fácil e rápida.

Apesar de grande parte dos integrantes da equipe não possuir experiência com 
o Figma\cite{figma}, tudo ocorreu de acordo com o esperado e os stakeholders se mostraram 
bem satisfeitos com o resultado que obtivemos, porém indicaram algumas alterações 
que gostariam que fossem aplicadas aos protótipos que foram feitos assim que 
possível.